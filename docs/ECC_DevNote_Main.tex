\documentclass[12pt,openany]{book}

% Preamble with packages, custom commands, etc.
\usepackage{commath, amsmath, amsthm}
\usepackage{polynomial}
\usepackage{enumerate}
\usepackage{soul} % highlight
\usepackage{lipsum}  % Just for generating text

% Theorem
\newtheorem{axiom}{Axiom}[chapter]
\newtheorem{theorem}{Theorem}[chapter]
\newtheorem{proposition}[theorem]{Proposition}
\newtheorem{corollary}{Corollary}[theorem]
\newtheorem{lemma}[theorem]{Lemma}

\theoremstyle{definition}
\newtheorem{definition}{Definition}[chapter]
\newtheorem{remark}{Remark}[chapter]
\newtheorem{exercise}{Exercise}[chapter]
\newtheorem{example}{Example}[chapter]
\newtheorem*{note}{Note}

% Colors
\usepackage[dvipsnames,table]{xcolor}
\definecolor{titleblue}{RGB}{0,53,128}
\definecolor{chaptergray}{RGB}{140,140,140}
\definecolor{sectiongray}{RGB}{180,180,180}

\definecolor{thmcolor}{RGB}{231, 76, 60}
\definecolor{defcolor}{RGB}{52, 152, 219}
\definecolor{lemcolor}{RGB}{155, 89, 182}
\definecolor{corcolor}{RGB}{46, 204, 113}
\definecolor{procolor}{RGB}{241, 196, 15}
\definecolor{execolor}{RGB}{90, 128, 127}

\definecolor{comments}{HTML}{6A9955} % A kind of forest green.
\definecolor{keyword}{HTML}{569CD6} % A medium blue..
\definecolor{string}{HTML}{CE9178} % A reddish-brown or terra cotta.
\definecolor{function}{HTML}{DCDCAA} % A beige or light khaki.
\definecolor{number}{HTML}{B5CEA8} % A muted green.
\definecolor{type}{HTML}{4EC9B0} %  A turquoise or teal.
\definecolor{class}{HTML}{4EC9B0} % Similar to types, a turquoise or teal.

\definecolor{aesblue}{RGB}{30,144,255}
\definecolor{aesgreen}{RGB}{0,128,0}
\definecolor{aesred}{RGB}{255,69,0}
\definecolor{aesgray}{RGB}{112,128,144}

% Fonts
\usepackage[T1]{fontenc}
\usepackage[utf8]{inputenc}
\usepackage{newpxtext,newpxmath}
\usepackage{sectsty}
\allsectionsfont{\sffamily\color{titleblue}\mdseries}

% Page layout
\usepackage{geometry}
\geometry{a4paper,left=.8in,right=.6in,top=.75in,bottom=1in,heightrounded}
\usepackage{fancyhdr}
\fancyhf{}
\fancyhead[LE,RO]{\thepage}
\fancyhead[LO]{\nouppercase{\rightmark}}
\fancyhead[RE]{\nouppercase{\leftmark}}
\renewcommand{\headrulewidth}{0.5pt}
\renewcommand{\footrulewidth}{0pt}

% Chapter formatting
\usepackage{titlesec}
\titleformat{\chapter}[display]
{\normalfont\sffamily\Huge\bfseries\color{titleblue}}{\chaptertitlename\ \thechapter}{20pt}{\Huge}
\titleformat{\section}
{\normalfont\sffamily\Large\bfseries\color{titleblue!100!gray}}{\thesection}{1em}{}
\titleformat{\subsection}
{\normalfont\sffamily\large\bfseries\color{titleblue!75!gray}}{\thesubsection}{1em}{}

% Table of contents formatting
\usepackage{tocloft}
\renewcommand{\cftchapfont}{\sffamily\color{titleblue}\bfseries}
\renewcommand{\cftsecfont}{\sffamily\color{chaptergray}}
\renewcommand{\cftsubsecfont}{\sffamily\color{sectiongray}}
\renewcommand{\cftchapleader}{\cftdotfill{\cftdotsep}}

% TikZ
\usepackage{tikz}
\usetikzlibrary{math} % for calculations
\usetikzlibrary{matrix, positioning, arrows.meta, calc, shapes.geometric, shapes.multipart, chains}
\usetikzlibrary{decorations.pathreplacing,calligraphy}

\usepackage{pgfplots}
\usepgfplotslibrary{fillbetween}
\pgfplotsset{compat=1.15}

% Pseudo-code
\usepackage[linesnumbered,ruled]{algorithm2e}
\usepackage{algpseudocode}
\usepackage{setspace}
\SetKwComment{Comment}{/* }{ */}
\SetKwProg{Fn}{Function}{:}{end}
\SetKw{End}{end}
\SetKw{DownTo}{downto}

% Define a new environment for algorithms without line numbers
\newenvironment{algorithm2}[1][]{
	% Save the current state of the algorithm counter
	\newcounter{tempCounter}
	\setcounter{tempCounter}{\value{algocf}}
	% Redefine the algorithm numbering (remove prefix)
	\renewcommand{\thealgocf}{}
	\begin{algorithm}
	}{
	\end{algorithm}
	% Restore the algorithm counter state
	\setcounter{algocf}{\value{tempCounter}}
}


% Listings
\usepackage{listings}
\renewcommand{\lstlistingname}{Code}%
\lstdefinestyle{C}{
	language=C,
	backgroundcolor=\color{white},
	basicstyle=\ttfamily\color{black},
	commentstyle=\color{green!70!black},
	keywordstyle={\bfseries\color{purple}},
	keywordstyle=[2]{\bfseries\color{red}},
	keywordstyle=[3]{\bfseries\color{type}},
	keywordstyle=[4]{\bfseries\color{JungleGreen}},
	keywordstyle=[5]{\bfseries\color{Magenta}},
	keywordstyle=[6]{\bfseries\color{RoyalBlue}},
	keywordstyle=[7]{\bfseries\color{Turquoise}},
	otherkeywords={bool, sscanf, inline},
	morekeywords=[2]{;},
	morekeywords=[3]{i8, i32, i64, u8,u32,u64, size\_t, FILE},
	morekeywords=[4]{
		rdtsc, measure\_cycle, measure\_time
	},
	morekeywords=[5]{
		MOVS\_LEA128CBC\_KAT\_TEST,
		create\_LEA128CBC\_KAT\_ReqFile,
		create\_LEA128CBC\_KAT\_FaxFile,
		create\_LEA128CBC\_KAT\_RspFile
	},
	morekeywords=[6]{false, true, MAX, MIN, INITIAL\_BUF\_SIZE},
	morekeywords=[7]{
		CryptoData
	},
	numberstyle=\tiny\color{gray},
	stringstyle=\color{purple},
	showstringspaces=false,
	breaklines=true,
	frame=single,
	framesep=3pt,
	%frameround=tttt,
	framexleftmargin=3pt,
	numbers=left,
	numberstyle=\small\color{gray},
	xleftmargin=15pt, % Increase the left margin
	xrightmargin=5pt,
	tabsize=4,
	belowskip=0pt,
	aboveskip=4pt
}
\lstdefinestyle{zsh}{
	language=bash,                  % Set the language to bash (closest to Zsh)
	backgroundcolor=\color{backColor},
	commentstyle=\color{commentColor}\ttfamily,
	keywordstyle=\color{keywordColor}\bfseries,
	stringstyle=\color{stringColor}\ttfamily,
	showspaces=false,               % Don't show spaces as underscores
	showstringspaces=false,         % Don't highlight spaces in strings
	breaklines=true,                % Automatic line breaking
	frame=none,                     % No frame around the code
	basicstyle=\ttfamily\color{white}, % White basic text color for contrast
	extendedchars=true,             % Allow extended characters
	captionpos=b,                   % Caption-position at bottom
	escapeinside={\%*}{*)},         % Allow LaTeX inside your code
	morekeywords={echo,ls,cd,pwd,exit,clear,man,unalias,zsh,source}, % Add more keywords
	upquote=true,                   % Ensure straight quotes are used
	literate={\$}{{\textcolor{red}{\$}}}1
	{:}{{\textcolor{red}{:}}}1
	{~}{{\textcolor{red}{\textasciitilde}}}1, % Color certain characters
}

% Usage: \lstinputlisting[style=zsh]{sourcefile.sh}
% or \begin{lstlisting}[style=zsh] ... \end{lstlisting}


% Table
\usepackage{booktabs}
\usepackage{tabularx}
\usepackage{multicol}
\usepackage{multirow}
\usepackage{array}
\usepackage{longtable}

% Hyperlinks
\usepackage{hyperref}
\hypersetup{
	colorlinks=true,
	linkcolor=titleblue,
	filecolor=black,      
	urlcolor=titleblue,
}

%Ceiling and Floor Function
\usepackage{mathtools}
\DeclarePairedDelimiter{\ceil}{\lceil}{\rceil}
\DeclarePairedDelimiter{\floor}{\lfloor}{\rfloor}


\newcommand{\mathcolorbox}[2]{\colorbox{#1}{$\displaystyle #2$}}

\newcommand{\N}{\mathbb{N}}
\newcommand{\Z}{\mathbb{Z}}
\newcommand{\Q}{\mathbb{Q}}
\newcommand{\R}{\mathbb{R}}
\newcommand{\C}{\mathbb{C}}
\newcommand{\F}{\mathbb{F}}
\newcommand{\zero}{\textcolor{red}{\texttt{0}}}
\newcommand{\one}{\textcolor{red}{\texttt{1}}}
\newcommand{\binaryfield}{\set{\zero,\one}}

\newcommand{\ie}{\textnormal{i.e.}}
\newcommand{\sol}{\textcolor{magenta}{\bf Solution.\ }}

\newcommand{\yes}{\textcolor{blue}{O}}
\newcommand{\no}{\textcolor{red}{X}}

\newcommand{\adjacent}{\parallel}

\newcommand{\src}{\mathsf{src}}
\newcommand{\dsc}{\mathsf{dsc}}
\newcommand{\state}{\mathsf{state}}
\newcommand{\data}{\mathsf{data}}

\begin{document}	
	% Title page
	\begin{titlepage}
		\begin{center}
			{\Huge\textsf{\textbf{Elliptic Curve Cryptograph}}\par}
			{\Large\textsf{\textbf{- Learning ECC -}}\par}
			\vspace{0.5in}
			{\Large {Ji Yong-Hyeon}\par}
			\vspace{1in}
%			\includegraphics[scale=.6]{images/AES_Encryption_Round.png}\par
			\vspace{1in}
			{\large\bf \textsf{Department of Information Security, Cryptology, and Mathematics}\par}
			{\textsf{College of Science and Technology}\par}
			{\textsf{Kookmin University}\par}
			\vspace{.25in}
			{\large \textsf{\today}\par}
		\end{center}
	\end{titlepage}
	
	\frontmatter
	\section*{Acknowledgements}

%	\section*{Abstract}

This book is about...

% Your abstract here

	
	\newpage
	\tableofcontents
	
	\newpage
	\mainmatter
	\chapter{Prime Field Operations}
\iffalse
\section{Time Measure in Linux}

\begin{lstlisting}[style=C]
#define _POSIX_C_SOURCE 200809L
#include <time.h>

typedef int8_t i8;
typedef int32_t i32;
typedef int64_t i64;

typedef uint8_t u8;
typedef uint32_t u32;
typedef uint64_t u64;

inline u64 rdtsc(void) {
	u32 lo, hi;
	
	__asm__ __volatile__ (
	"rdtsc" : "=a" (lo), "=d" (hi)
	);
	
	return ((u64)hi << 32) | lo;
}

u64 measure_cycle(void (*func)(u8*, u8*), u8* para1, u8* para2) {
	u64 start, end;
	const u64 num_runs = 10000;
	
	func(para1, para2);
	start = rdtsc();
	for (i32 i = 0; i < num_runs; i++)
		func(para1, para2);
	end = rdtsc();
	
	return (end - start) / num_runs;
}

double measure_time(void (*func)(u8*, u8*), u8* para1, u8* para2) {
	struct timespec start, end;
	double cpu_time_used;
	const double num_runs = 10000;
	
	func(para1, para2);
	clock_gettime(CLOCK_MONOTONIC, &start);
	for (i32 i = 0; i < num_runs; i++)
		func(para1, para2);
	clock_gettime(CLOCK_MONOTONIC, &end);
	
	cpu_time_used =
		(end.tv_sec - start.tv_sec) +
		(end.tv_nsec - start.tv_nsec) / 1e9;
	
	return cpu_time_used / num_runs;
}
\end{lstlisting}
\fi
\section{Data Representation}

\begin{table}[h!]\centering\renewcommand{\arraystretch}{1.25} % Increase row height by 1.5 times
	\begin{tabular}{@{\extracolsep{\fill}}>{\bfseries}l||c|c|c|c|c|c|c|c|c|c|c|c}
		\toprule[1.2pt]
		128-bit Input String & \multicolumn{12}{c}{\texttt{0x77FDDC58464B01FC6606BC465BF5CBCB}} \\
		\hline
		String Index & \cellcolor{red!20}0 & \cellcolor{red!20}$\cdots$ & \cellcolor{red!20}7 & \cellcolor{green!20}8 & \cellcolor{green!20}$\cdots$ & \cellcolor{green!20}15 & \cellcolor{blue!20}16 & \cellcolor{blue!20}$\cdots$ & \cellcolor{blue!20}23 & \cellcolor{orange!20}24 & \cellcolor{orange!20}$\cdots$ & \cellcolor{orange!20}31 \\
		\hline
		\multirow{2}{*}{Split into Words} & \multicolumn{3}{c}{\texttt{77FDDC58}} & \multicolumn{3}{c}{\texttt{464B01FC}} & \multicolumn{3}{c}{\texttt{6606BC46}} & \multicolumn{3}{c}{\texttt{5BF5CBCB}} \\
		& \multicolumn{3}{c}{$\data[3]$} & \multicolumn{3}{c}{$\data[2]$} & \multicolumn{3}{c}{$\data[1]$} & \multicolumn{3}{c}{$\data[0]$}\\
		\hline
		\multirow{4}{*}{$\data[0]$} & \multicolumn{12}{c}{\texttt{(data[0] = '5'-'0';)} -> \texttt{(data[0]} $\ll=$ 4;)} \\
		& \multicolumn{12}{c}{-> \texttt{(data[0] |= 'B'-'0';)} -> \texttt{(data[0]} $\ll=$ 4;)}\\
		& \multicolumn{12}{c}{-> \texttt{(data[0] |= 'F'-'0';)} -> $\cdots$}\\
		\cline{2-13}
		& \multicolumn{12}{c}{\texttt{0x5 -> 0x50 -> 0x5B -> 0x5B0 -> $\cdots$}}\\
		\bottomrule[1.2pt]
	\end{tabular}
\end{table}

\section{NIST P256}
Define a prime number \begin{align*}
p_{256} &= 2^{256}-2^{224}+2^{192}+2^{96}-1 \\
&=2^{32*8}-2^{32*7}+2^{32*6}+2^{32*3}-2^{32*0}\\
&= 2^{64*4}-2^{64*(7/2)}+2^{64*3}+2^{64*(3/2)}-2^{64*0}
\end{align*} that is used in the context of cryptography, particularly in the construction of elliptic curves for cryptographic purposes.
For prime $p$, let \begin{align*}
	m = \ceil*{\log_2 p},\quad t=\ceil*{\frac{m}{W}}.
\end{align*} For example, for $p=2^{256}$, we have $t=\ceil*{\frac{\ceil{\log_2 2^{256}}}{2^{32}}}$. Note that \begin{itemize}
\item $A = A[t-1]\adjacent\cdots\adjacent A[2]\adjacent A[1]\adjacent A[0]$
\item $a = 2^{(t-1)W}A[t-1]$+$\cdots 2^{2W}A[2]+2^{W}A[1]+A[0]$
\end{itemize}

\begin{lstlisting}[style=C]
#ifdef _64BIT_SYSTEM
typedef u64 field_element[4]; // For 64-bit systems
#else
typedef u32 field_element[8]; // For 32-bit systems
#endif

// Example for modular addition (simplified)
void mod_add(field_element a, field_element b, field_element result) {
	uint64_t carry = 0;
	for (int i = 0; i < 4; ++i) { // Assuming 64-bit system
		uint64_t temp = (uint64_t)a[i] + b[i] + carry;
		result[i] = temp & 0xFFFFFFFFFFFFFFFF; // Keep only 64 bits
		carry = temp >> 64; // Carry for the next iteration
	}
	
	// Modular reduction if necessary
	if (carry || is_greater_or_equal(result, p256)) {
		// Subtract p256 if result >= p256
		subtract_p256(result);
	}
}

void subtract_p256(field_element x) {
	// This is a simplified version. In practice, you'd need to handle underflows.
	// Subtract (2^256 - 2^224 + 2^192 + 2^96 - 1)
	// In practice, implement this function based on the specific structure of p256
	// and considering the binary representation of the field elements.
}
\end{lstlisting}

\newpage
\section{Multi-precision Addition}

\begin{algorithm}[H]
	\DontPrintSemicolon
	\caption{Multi-Precision Addition}
	\KwIn{$u,v\intco{0,2^{Wt}}\subseteq\Z$}
	\KwOut{$(\varepsilon,w)$ where $w=u+v\bmod 2^{Wt}$ and $\varepsilon\in\set{0,1}$ is carry bit}
	\BlankLine
	$(\varepsilon,W[0])\gets U[0]+V[0]$\;
	\For{$i=1$ \KwTo $t-1$}{
		$(\varepsilon,W[i])\gets U[i] + V[i] + \varepsilon$\;
	}
	\Return $(\varepsilon, w)$\;
\end{algorithm}
\begin{example}
\ \begin{table}[h!]\centering\renewcommand{\arraystretch}{1.25}
	{\ttfamily\begin{tabular*}{\textwidth}{@{\extracolsep{\fill}}cccccc}
	$U$ & & 0xFFFFFFFF & 0xFFFFFFFF & 0xFFFFFFFF & 0xFFFFFFFF \\
	$V$ & + & 0xFFFFFFFF & 0xFFFFFFFF & 0xFFFFFFFF & 0xFFFFFFFF \\
	$\varepsilon$ & 1 & 1 & 1 & 1 & 0 \\ \cline{2-6}
	$W$ & 0x00000001 & 0xFFFFFFFF & 0xFFFFFFFF & 0xFFFFFFFF & 0xFFFFFFFE \\
	\end{tabular*}}
\end{table}	
\end{example}


 % Prime Field Operations
	\chapter{Elliptic Curve Theory}

\section{A Puzzle of Squares and Pyramids}

Consider the following question:\begin{quote}
``What is the number of balls that may be piled as a square pyramid and also re-arranged into a square array?''
\end{quote}
To address this, let \( x \) be the height of the pyramid. The number of balls in a pyramid of height \( x \) is given by:
\[
1^2 + 2^2 + 3^2 + \ldots + x^2 = \frac{x(x+1)(2x+1)}{6}
\]
We seek a configuration where this sum also forms a perfect square, i.e.,
\[
y^2 = \frac{x(x+1)(2x+1)}{6}
\]
This equation forms the basis of our puzzle, intertwining the concepts of geometric and numeric squares.
\vspace{24pt}
\begin{center}
	\begin{tikzpicture}[scale=1]
		\begin{axis}[
			axis lines=middle,
			xlabel=$x$,
			ylabel=$y$,
			ymin=-1.25, ymax=1.25,
			xmin=-1.25, xmax=1.25,
			domain=-1.25:1.25,
			samples=1000,
			grid=both,
			grid style={line width=.1pt, draw=gray!10},
			major grid style={line width=.1pt,draw=gray!50},
			]
			\addplot [red, smooth, thick, name path=A] {sqrt(x*(x+1)*(2*x+1)/6)};
			\addplot [red, smooth, thick, name path=B] {-sqrt(x*(x+1)*(2*x+1)/6)};
			%		\addplot [fill=blue!10] fill between[of=A and B];
		\end{axis}
	\end{tikzpicture}
\end{center}

\newpage
\subsection{Diophantus' Approach}

We consider a set of known points to produce new points. The trivial solutions \( (0,0) \) and \( (1,1) \) fit the equation of the line \( y = x \). 
\begin{center}
	\begin{tikzpicture}[scale=1]
		\begin{axis}[
			axis lines=middle,
			xlabel=$x$,
			ylabel=$y$,
			ymin=-1.25, ymax=1.25,
			xmin=-1.25, xmax=1.25,
			domain=-1.25:1.25,
			samples=1000,
			grid=both,
			grid style={line width=.1pt, draw=gray!10},
			major grid style={line width=.1pt,draw=gray!50},
			]
			\addplot [red, smooth, thick, name path=A] {sqrt(x*(x+1)*(2*x+1)/6)};
			\addplot [red, smooth, thick, name path=B] {-sqrt(x*(x+1)*(2*x+1)/6)};
			\addplot [blue, smooth, thick, name path=A] {x};
			
			\addplot[mark=*, mark options={fill=red, scale=1.5}] coordinates {(1,1)};
			\node[above left] at (axis cs:1,1) {(1,1)};
			\addplot[mark=*, mark options={fill=red, scale=1.5}] coordinates {(0,0)};
			\node[above left] at (axis cs:0,0) {(0,0)};
			
			\addplot[mark=x, mark options={fill=blue, draw=blue, scale=3, line width=2pt}] coordinates {(1/2,1/2)};
			\node[below right] at (axis cs:1/2,1/2) {($\frac{1}{2}$,$\frac{1}{2}$)};
			%		\addplot [fill=blue!10] fill between[of=A and B];
		\end{axis}
	\end{tikzpicture}
\end{center}

Intersecting this line with the curve described by our pyramid problem, we rearrange terms:
\begin{align*}
\frac{x(x+1)(2x+1)}{6}&=x^2,\\
(x^2+x)(2x+1)&=6x^2,\\
2x^3+x^2+2x^2+x&=6x^2,\\
x(2x^2-3x+1)&=0,\\
x(x-1)(2x-1)&=0.
\end{align*}

We find that \( x = \frac{1}{2} \) is a solution, implying \( y = \frac{1}{2} \). The symmetry of the curve also yields \( \left(\frac{1}{2}, -\frac{1}{2}\right) \) as another solution.

\subsection{Extending Diophantus' Method}

Consider the line through \( \left(\frac{1}{2}, -\frac{1}{2}\right) \) and \( (1,1) \), which implies \( y = 3x - 2 \). 
\begin{center}
	\begin{tikzpicture}[scale=1]
		\begin{axis}[
			axis lines=middle,
			xlabel=$x$,
			ylabel=$y$,
			ymin=-1.25, ymax=1.25,
			xmin=-1.25, xmax=1.25,
			domain=-1.25:1.25,
			samples=1000,
			grid=both,
			grid style={line width=.1pt, draw=gray!10},
			major grid style={line width=.2pt,draw=gray!50},
			]
			\addplot [red, smooth, thick, name path=A] {sqrt(x*(x+1)*(2*x+1)/6)};
			\addplot [red, smooth, thick, name path=B] {-sqrt(x*(x+1)*(2*x+1)/6)};
			\addplot [blue, smooth, thick, name path=A] {3*x-2};
			
			\addplot[mark=*, mark options={fill=green, scale=1.5}] coordinates {(1,1)};
			\node[above left] at (axis cs:1,1) {(1,1)};
			\addplot[mark=*, mark options={fill=green, scale=1.5}] coordinates {(1/2,-1/2)};
			\node[below right] at (axis cs:1/2,-1/2) {($\frac{1}{2}$,$-\frac{1}{2}$)};
			
			%		\addplot [fill=blue!10] fill between[of=A and B];
		\end{axis}
	\end{tikzpicture}
\end{center}

Intersecting this with our curve, we derive:

\begin{equation}
	x^3 - \frac{51}{2}x^2 + \ldots = 0
\end{equation}

This leads to the solutions \( x = 24 \) and \( y = 70 \), demonstrating the power of algebraic manipulation and geometric insight.

\newpage
\section{Why is it called an Elliptic Curve?}
%\begin{tikzpicture}
%	% Axes
%	\draw[->] (-3,0) -- (3,0) node[right] {$w$};
%	\draw[->] (0,-3) -- (0,3) node[above] {$y$};
%	
%	% Sine function
%	\draw[domain=-3:3, smooth, variable=\x, blue] plot ({\x}, {sin(deg(\x))});
%	\node[blue] at (2,1.5) {$y = \sin w$};
%	
%	% Inverse Sine function (integral representation)
%	\fill[red, opacity=0.3] (0,0) -- plot[domain=0:0.5] ({\x}, {sin(deg(\x))}) -- (0.5,0) -- cycle;
%	\node[red] at (1,-1) {$w(y) = \sin^{-1} y$};
%	
%	% Abel's function
%	\draw[domain=-0.99:0.99, smooth, variable=\x, green] plot ({\x*3}, {3*sqrt(1-\x*\x)});
%	\node[green] at (-2,2.5) {$F(w)$};
%\end{tikzpicture}

The term ``elliptic curve'' has its roots in the quest to measure the circumference of an ellipse. Consider the trigonometric function \( y = \sin w \). The inverse function, \( w(y) = \sin^{-1} y \), is expressed as an integral:
\[
w(y) = \sin^{-1} y = \int_{0}^{y} \frac{1}{\sqrt{1 - t^2}} \, dt
\]
This integral is foundational in understanding the link between elliptic curves and elliptic integrals.

\subsection{Abel's Insight}
Niels Henrik Abel, a prominent mathematician, extended this concept. Starting with \( y = \sin w \), Abel explored the inverse functions of elliptic integrals, uncovering their double periodicity. He defined the function:
\[
F(w) = \int_{0}^{w} \frac{dz}{\sqrt{(1 - z^2)(1 - k^2 z^2)}}
\]
Abel's work laid the groundwork for understanding the complex nature of elliptic curves.

\subsection{The Geometry of an Ellipse}

An ellipse is defined by the equation \( x^2/a^2 + y^2/b^2 = 1 \). This simple equation belies the complexity of calculating its arc length.

\subsection{The Arc Length of an Ellipse}

Defining \( k^2 = 1 - \frac{b^2}{a^2} \) and changing variables \( x \rightarrow ax \), we express the arc length of an ellipse as:
\[
a \int_{-1}^{1} \sqrt{\frac{1 - k^2 x^2}{1 - x^2}} \, dx = a \int_{-1}^{1} \frac{1 - k^2 x^2}{\sqrt{(1 - x^2)(1 - k^2 x^2)}} \, dx
\]
This leads to the following representation of the arc length:
\[
\textbf{Arc Length} = a \int_{-1}^{1} \frac{1 - k^2 x^2}{y} \, dx\quad\text{with}\quad y^2=(1-x^2)(1-k^2x).
\]

\subsection{Connecting to an Elliptic Curve}

The ellipse's arc length calculation brings us to a critical realization. An elliptic integral is generally expressed as:
\[
\int R(x,y) \, dx
\]
This integral, deeply connected to the geometry of ellipses, underpins the theory of elliptic curves.

\newpage
\section*{Double Periodicity in Elliptic Curves}

Elliptic curves are intimately connected with the study of complex tori, which can be represented through the use of doubly periodic functions. A fundamental example of such a function is the Weierstrass $\wp$ function, defined by a lattice $\Lambda$ in the complex plane. 

\subsection*{The Weierstrass $\wp$ Function}

Given a lattice $\Lambda \subset \mathbb{C}$, the Weierstrass $\wp$ function is defined as:
\begin{equation}
	\wp(z; \Lambda) = \frac{1}{z^2} + \sum_{\omega \in \Lambda \setminus \{0\}} \left( \frac{1}{(z - \omega)^2} - \frac{1}{\omega^2} \right).
\end{equation}

This function is even, $\wp(-z) = \wp(z)$, and exhibits double periodicity with respect to the lattice $\Lambda$, meaning:
\begin{equation}
	\wp(z + \omega) = \wp(z) \quad \text{for all} \, \omega \in \Lambda.
\end{equation}

\subsection*{Elliptic Curves and the $\wp$ Function}

An elliptic curve can be associated with the Weierstrass $\wp$ function. Specifically, an elliptic curve over $\mathbb{C}$ can be described in the Weierstrass form:
\begin{equation}
	y^2 = 4x^3 - g_2x - g_3,
\end{equation}
where $g_2$ and $g_3$ are constants derived from the lattice $\Lambda$. The coordinates $(x, y)$ on the elliptic curve correspond to the values of the Weierstrass $\wp$ function and its derivative:
\begin{equation}
	x = \wp(z; \Lambda), \quad y = \wp'(z; \Lambda).
\end{equation}


\section*{Double Periodicity}

\subsection*{Two linearly independent periods.}
\[
\phi(z + w_1) = \phi(z + w_2) = \phi(z) \text{ for all complex number } z.
\]

\subsection*{It satisfies}
\[
[\phi'(z)]^2 = 4\phi(z)^3 - 60 G_4 \phi(z) - 140 G_6
\]

\begin{itemize}
	\item So for $x=\phi(z)$ and $y=\phi'(z)$
	\item $y^2=4x^3-60G_3x-140G_6$
\end{itemize}


\newpage
\section*{Elliptic Functions and Elliptic Curves}

The \(\wp\)-function and its derivative satisfy an algebraic relation
\[
\wp'(z)^2 = \wp(z)^3 + A \wp(z) + B
\]

The double periodicity means that it is a function on the quotient space \(\mathbb{C}/\Lambda\), where \(\Lambda\) is the lattice
\[
\Lambda = \{ n_1\omega_1 + n_2\omega_2 : n_1,n_2\in\mathbb{Z}\}.
\]
\vspace{8pt}
\begin{center}
\begin{tikzpicture}[scale=1]
	\fill[yellow!50] (-4,-3) -- (4,-1) -- (5,3) -- (-3,1) -- cycle;
	\draw[thin,gray!40] (-5,-4) grid (6,4);
	% Draw axes
%	\draw[->] (-2,0) -- (4,0) node[right] {$\Re$}; % Real axis
%	\draw[->] (0,-2) -- (0,4) node[above] {$\Im$}; % Imaginary axis
	
	% Define lattice points
	\foreach \x in {-5,...,6} \foreach \y in {-4,...,4} {
		\fill (\x,\y) circle (2pt); % Lattice points
	}
	
	\draw[blue,thick,->] (-4,-3) -- (4,-1) node[right] {$\omega_1$};
	\draw[red,thick,->] (-4,-3) -- (-3,1) node[right] {$\omega_2$};
	\draw[green!50!black,thick,->] (-4,-3) -- (5,3) node[right] {$\omega_1+\omega_2$};
	 \node[align=center] at (1,-5) {The lattice $L$ is generated by $\omega_1$ and $\omega_2$\\in the quotient space $\mathbb{C}/L$.};
\end{tikzpicture}
\end{center}
\section*{Elliptic Functions and Elliptic Curves}

Elliptic functions and elliptic curves are fundamental objects in complex analysis and algebraic geometry, respectively. They are interconnected through the Weierstrass $\wp$ function and its properties.

\subsection*{Weierstrass Elliptic Functions}

The Weierstrass elliptic functions are defined with respect to a lattice $\Lambda \subset \mathbb{C}$. The Weierstrass $\wp$ function, a key example of an elliptic function, is defined as:
\begin{equation}
	\wp(z; \Lambda) = \frac{1}{z^2} + \sum_{\omega \in \Lambda \setminus \{0\}} \left( \frac{1}{(z - \omega)^2} - \frac{1}{\omega^2} \right).
\end{equation}
This function is doubly periodic and meromorphic with poles of order two at lattice points.

\subsection*{Elliptic Curves and the Weierstrass $\wp$ Function}

An elliptic curve can be described as a set of points satisfying a cubic equation in two variables. Over the complex numbers, this curve can be associated with the Weierstrass $\wp$ function.

An elliptic curve in Weierstrass form is given by:
\begin{equation}
	y^2 = 4x^3 - g_2x - g_3,
\end{equation}
where $g_2$ and $g_3$ are constants determined by the lattice $\Lambda$. The function $\wp$ and its derivative relate to the curve as follows:
\begin{align}
	x &= \wp(z; \Lambda), \\
	y &= \wp'(z; \Lambda).
\end{align}
This establishes a correspondence between points on the complex torus $\mathbb{C}/\Lambda$ and points on the elliptic curve.

\subsection*{Properties of Elliptic Curves}

Elliptic curves have several important properties:

\begin{itemize}
	\item They form a group under a geometrically defined addition operation.
	\item The addition operation on the curve corresponds to the addition of points in the complex plane modulo the lattice $\Lambda$.
	\item Elliptic curves over finite fields have applications in number theory and cryptography.
\end{itemize}


\newpage
\section*{The Complex Points on an Elliptic Curve}

The \(\phi\)-function gives a complex analytic isomorphism
\[
\frac{\mathbb{C}}{L} = \big(\phi(z),\phi'(z)\big) \rightarrow E(\mathbb{C})
\]
with the notation that \( \mathbb{C} \) is the complex numbers, \( L \) is a lattice, and \( E(\mathbb{C}) \) is the set of complex points on an elliptic curve.

%\begin{figure}[h!]
%	\centering
%	\includegraphics[width=0.5\textwidth]{parallelogram_to_torus.png}
%	\caption{Parallelogram with opposite sides identified = a torus.}
%\end{figure}

Thus the points of \( E \) with coordinates in the complex numbers \( \mathbb{C} \) form a \textit{torus}, that is, the surface of a donut.

%\begin{figure}[h!]
%	\centering
%	\includegraphics[width=0.5\textwidth]{torus.png}
%	\caption{The complex points \( E(\mathbb{C}) \) forming a torus.}
%\end{figure}

\section*{X\(^2\) + Y\(^2\) = C}

\begin{itemize}
	\item Let \( x = a + b\sqrt{-1} \), \( y = c + d\sqrt{-1} \).
	\item The solution over complex numbers is a surface, in fact topologically sphere.
	\item If unbelievable, check out level curves.
	\item Furthermore, it has group structure.
	\[ (a + b\sqrt{-1})(c + d\sqrt{-1}) \text{ becomes } ac-bd + (ad+bc)\sqrt{-1} \]
\end{itemize}

\section*{Why is it called Torus?}
\begin{itemize}
	\item Complex Tori \[
	y^2=x(x^2-1)
	\]
	\item If we introduce \textit{points at infinity} and the \textit{complex numbers}, we can argue that the graph is a torus.
\end{itemize}

\section*{Why Elliptic Curve?}

\begin{itemize}
	\item Discrete Logarithm Problem
	\item Given a finite group \( G \) with two of its elements \( a \) and \( b \).
	\item Find an integer \( x \) such that, \( a^x = b \) if it exists.
	\item Example: Non-zero elements of some finite field.
\end{itemize}

\section*{Better groups?}

For a finite field \( F \),
\[
GL_2(F) = \left\{ \begin{pmatrix}
	a & b \\
	c & d
\end{pmatrix} \mid ad - bc \neq 0, \, a,b,c,d \in F \right\}
\]

\begin{itemize}
	\item The Times(London) Jan. 1999 An Irish schoolgirl Sarah Flannery used matrices as an alternative to RSA. Her algorithm is far faster than the RSA and equally secure.
	\item The Art of Computer Programming by Donald Knuth
\end{itemize}

How about this group?

\begin{itemize}
	\item \( F = \mathbb{Z} / 17\mathbb{Z} = \mathbb{Z} \text{ (mod 17)} \)
	\item \( 6^2 = 36 \equiv 2 \text{ mod 17} \)
	\item \( 6 \text{ behaves like } \sqrt{2} \)
\end{itemize}

\( X^2 - 2Y^2 = 1 \)

\( (3 + 2 \sqrt{2})(3 - 2 \sqrt{2}) = 1 \)

\( (3 + 12)(3 - 12) = -36 \equiv 1 \text{ mod 17} \)



\begin{itemize}
	\item 
	Let \( G = \{(x, y) \mid x^2 - 2y^2 = 1 \text{ over } \mathbb{F}\} \) The operation on \( G \) is defined as:
	\[
	(x_1, y_1) \cdot (x_2, y_2) =
	\]
	\[
	\left( x_1 + \sqrt{2} y_1 \right) \left( x_2 + \sqrt{2} y_2 \right) =
	\]
	\[
	= \left( x_1 x_2 + 2y_1 y_2 \right) + \sqrt{2} \left( x_1 y_2 + x_2 y_1 \right)
	\]
	\[
	(x_1, y_1) \cdot (x_2, y_2) =
	\]
	\[
	= \left( x_1 x_2 + 2y_1 y_2 \right) x_1 y_2 + x_2 y_1
	\]
\end{itemize}

\section*{Why Elliptic Curve?}

\begin{itemize}
	\item DLP (Discrete Logarithm Problem) on finite field can be solved faster than we thought!
	\item by ``index calculus''
	\item To protect against this attack...
	\item Elliptic curves!
\end{itemize}



\newpage
\begin{tikzpicture}[scale=.5, >=Stealth]
	% Grid
	\draw[thin,gray!40] (-10,-10) grid (10,10);
	% Axes
	\draw[thick,<->] (-10,0)--(10,0) node[right]{$x$};
	\draw[thick,<->] (0,-10)--(0,10) node[above]{$y$};
	% Points
	% Here you would place the points you want to plot, for example:
	 \node[draw,circle,inner sep=2pt,fill] at (1,2) {};
	 \node[draw,circle,inner sep=2pt,fill] at (2,3) {};
	 \node[draw,circle,inner sep=2pt,fill] at (3,6) {};
	
	% If you have integer solutions for x and y, you would add them here.
	% For instance, if you had a solution (x,y) = (2,4), you would add:
	 \node[draw,circle,inner sep=2pt,fill] at (2,4) {};
	
	% Labels
	% Similarly, label them like so (for the point (2,4)):
	 \node at (2,4) [below left] {$(2,4)$};
\end{tikzpicture}



	\chapter{Elliptic Curves in Cryptography}

\begin{itemize}
	\item Elliptic Curve (EC) cryptography were first proposed in 1985 independently by Neal Koblitz and Victor Miller.
	\item The \textbf{discrete logarithm} problem on elliptic curve groups is believed to be more difficult than the corresponding problem in the multiplicative group of non-zero elements of the underlying finite field.
\end{itemize}

\section*{On finite fields}

Consider \(y^2 \equiv x^3 + 2x + 3 \pmod{5}\)

\begin{align*}
	x = 0, &\ y^2 = 3 \quad &\text{no solution (mod 5)} \\
	x = 1, &\ y^2 = 6 \equiv 1, \quad &y = 1,4 \text{ (mod 5)} \\
	x = 2, &\ y^2 = 15 \equiv 0, \quad &y = 0 \text{ (mod 5)} \\
	x = 3, &\ y^2 = 36 \equiv 1, \quad &y = 1,4 \text{ (mod 5)} \\
	x = 4, &\ y^2 = 75 \equiv 0, \quad &y = 0 \text{ (mod 5)}
\end{align*}

Then points on the elliptic curve are \((1,1)\), \((1,4)\), \((2,0)\), \((3,1)\), \((3,4)\), \((4,0)\) and the point at infinity. Denote it by \( \mathcal{O} \).

\section*{Notation}

\begin{itemize}
	\item \( GF(q) \) or \( \mathbb{F}_q \): finite field with \( q \) elements, typically, \( q = p \) where \( p \) is prime, or \( 2^m \).
	\item \( E(\mathbb{F}_q) \): elliptic curve over \( \mathbb{F}_q \).
	\item \( (x, y) \): point on \( E(\mathbb{F}_q) \).
	\item \( \mathcal{O} \): point at infinity.
\end{itemize}

\newpage
\section*{Definition of Elliptic curves}

\begin{itemize}
	\item An elliptic curve over a field \( K \) is a non-singular cubic curve in two variables, \( f(x,y) = 0 \) with a rational point (which may be a point at infinity).
	
	\item The field \( K \) is usually taken to be the complex numbers, reals, rationals, algebraic extensions of rationals, \( p \)-adic numbers, or a \textit{finite field}.
	
	\item Elliptic curves groups for cryptography are examined with the underlying fields of \( \mathbb{F}_p \) (where \( p > 3 \) is a prime) and \( \mathbb{F}_{2^m} \) (a binary representation with \( 2^m \) elements).
\end{itemize}

\section*{EC}

An \textit{elliptic curve} is a plane curve defined by an equation of the form, when characteristic is neither 2 nor 3, and \ldots{} What the hell?

\[ y^2 = x^3 + ax + b \]

\section*{Hmm...}

\begin{itemize}
	\item \(x^3 + y^3 + 1 = 0\) is a cubic curve...?
	\item Let \(x = u + v\), \(y = u - v\).
	\item Then \( (u+v)^3 + (u-v)^3 + 1 = 0 \).
	\item This simplifies to \( 2u^3 + 6uv^2 + 1 = 0 \).
	\item Which leads to \( 6(v/u)^2 = -(1/u)^3 - 2 \).
	\item So \( X = -6/u\), \( Y = 36v/u \).
	\item Hence \( Y^2 = X^3 - 432 \).
\end{itemize}

\section*{Weierstrass Equation}

A two-variable equation \( F(x,y) = 0 \), forms a curve in the plane.

The generalized Weierstrass Equation of elliptic curves:
\[ y^2 + a_1xy + a_3y = x^3 + a_2x^2 + a_4x + a_6 \]

\section*{Quadratic Equation}

\begin{itemize}
	\item \( x^2 + ax + b = 0 \)
	\item \( x = t - \frac{a}{2} \)
	\item \( t^2 - \frac{a^2}{4} - 4b = 0 \)
\end{itemize}

\section*{Cubic Equation}

\begin{itemize}
	\item \( x^3 + ax^2 + bx + c = 0 \)
	\item \( x = t - \frac{a}{3} \)
	\item \( t^3 + pt + q = 0 \)
	\item \( p = b - \frac{a^2}{3} \)
	\item \( q = c + \frac{2a^3}{27} - \frac{ab}{3} \)
\end{itemize}

\section*{Field Characteristics}

\begin{itemize}
	\item If characteristic field is not 2:
	\begin{equation*}
		\left(v + \frac{a_1x}{2} + \frac{a_3}{2}\right)^2 = x^3 + \left(\frac{a_1^2}{4} + a_2\right)x^2 + a_4x + \left(\frac{a_1a_3}{4} + a_6\right)
	\end{equation*}
	\begin{equation*}
		\Rightarrow y_1^2 = x^3 + a_2'x^2 + a_4'x + a_6'
	\end{equation*}
	
	\item If characteristics of field is neither 2 nor 3:
	\begin{equation*}
		x_1 = x + \frac{a_2}{3}
	\end{equation*}
	\begin{equation*}
		\Rightarrow y_1^2 = x_1^3 + \Delta x + B
	\end{equation*}
\end{itemize}

\section*{Discriminant}

\begin{itemize}
	\item Discriminant of \( x^2 + bx + c \) is \( b^2 - 4c \)
	\item \( b^2 - 4c \) is non-zero \( \Leftrightarrow \) no double roots
	\item Discriminant of \( x^3 + ax + b \) is \( -4a^3 - 27b^2 \)
	\item \( -4a^3 - 27b^2 \) is non-zero \( \Leftrightarrow \) no double roots
\end{itemize}

\section*{j-invariant}

\begin{itemize}
	\item Define \( j \) of this elliptic curve \( E \) as \( j(E)/1728 = 4a^3/(4a^3 + 27b^2) \)
	\item If we change \( x = m^2x, y = m^3y \), get \( \tilde{E} \):
	\item then \( j(E) = j(\tilde{E}) \)
	\item \( j \)-value fixes \( E \)
\end{itemize}

\[ y^2 = x^3 + ax + b \]

\section*{j-invariant}

\begin{itemize}
	\item If we change \( x = m^2x, y = m^3y \), get \( \tilde{E} \):
	\item then \( j(E) = j(\tilde{E}) \)
	\item Why not something like \( x = mx + ny^2 + s \)?
	\item It has to keep the point at infinity and keep the form \( y^2 = x^3 + ax + b \)
\end{itemize}

\section*{Points on the Elliptic Curve (EC)}

\begin{itemize}
	\item Elliptic Curve over field \( L \)
	\item \( E(L) = \{\infty\} \cup \{(x, y) \in L \times L \mid y^2 + \ldots = x^3 + \ldots\} \)
	\item It is useful to add the point at infinity.
\end{itemize}

\section*{Group Law}

\begin{itemize}
	\item A group law may be defined where the sum of two points is the reflection across the x-axis of the third point on the same line
	\item Chords and tangents
\end{itemize}

\section*{The Abelian Group}

Given two points \( P, Q \) on \( E \), there is a third point, denoted by \( P+Q \) on \( \bar{E} \), and the following relations hold for all \( P, Q, R \) in \( E \).

\begin{itemize}
	\item \( P + Q = Q + P \) (commutativity)
	\item \( (P + Q) + R = P + (Q + R) \) (associativity)
	\item \( P + O = O + P = P \) (existence of an identity element)
	\item there exists \( (-P) \) such that \( (-P) + P = O \) (existence of inverses)
\end{itemize}

\section*{Associativity}

\begin{itemize}
	\item \( (P+Q)+R = P+(Q+R) \)
	\item Associativity is non-trivial.
	\item It gives Pascal's theorem and Pappus's theorem.
\end{itemize}

\section*{Elliptic Curve Picture}

\begin{itemize}
	\item Consider elliptic curve \( E: y^2 = x^3 - x + 1 \)
	\item If \( P_1 \) and \( P_2 \) are on \( E \), we can define \( P_3 = P_1 + P_2 \) as shown in the picture.
\end{itemize}

\section*{Doubling of a point}

\begin{itemize}
	\item Let \( P=Q \)
	\item \( 2y_1 \frac{dy}{dx} = 3x_1^2 + a \)
	\item \( m = \frac{dy}{dx} = \frac{3x_1^2 + a}{2y_1} \)
	\item If \( y_1 \neq 0 \) (since then \( P_1 + P_2 = \infty \)):
	\begin{itemize}
		\item \( 0 = x^3 - m^2x^2 + \ldots \)
		\item \( x_3 = m^2 - 2x_1, y_3 = m(x_1 - x_3) - y_1 \)
	\end{itemize}
	\item What happens when \( P_2 = \infty = O \)?
\end{itemize}


\section*{Sum of two points}

Define for two points \( P (x_1, y_1) \) and \( Q (x_2, y_2) \) in the Elliptic curve:

\[ \lambda = \begin{cases} 
	\frac{y_2 - y_1}{x_2 - x_1} & \text{for } x_1 \neq x_2 \\
	\frac{3x_1^2 + a}{2y_1} & \text{for } x_1 = x_2
\end{cases} \]

Then \( P + Q \) is given by \( R(x_3, y_3) \):

\[ x_3 = \lambda^2 - x_1 - x_2 \]
\[ y_3 = \lambda(x_3 - x_1) + y_1 \]

\section*{What is -P?}

\begin{itemize}
	\item \( y^2 = x^3 + ax + b \)
	\item \( P = (x_1, y_1) \)
	\item What is -P? Is -P = \( (x_1, -y_1) \)?
	\item Yes. But this works only for \( y^2 = x^3 + ax + b \).
	\item For \( y^2 + a_1xy + a_3y = x^3 + a_2x^2 + a_4x + a_6 \)
	\item -P = \( (x_1, -a_1x_1 - a_3 - y_1) \)
\end{itemize}

\section*{Motivation}

\begin{itemize}
	\item over \( \mathbb{F}_3 \)
	\item \( Y^2Z + 2XYZ + YZ^2 = X^3 - XZ^2 + 7Z^3 \) has a solution (1,2,1).
	\item Note that (0,1,0) is a solution.
	\item Important Point 1: We do not say (0,0,0) is a solution of the Weierstrass equation.
\end{itemize}

\section*{Homogeneous vs Affine}

\begin{itemize}
	\item Important Point 2: We treat \( (1,2,1) \sim (2,1,2) \), i.e., consider them to be identical and call it a point of the curve given by the Weierstrass equation.
	\item \( \frac{5^2}{13^2} + \frac{12^2}{13^2} = \frac{13^2}{13^2} \)
	\item \( \frac{10^2}{26^2} + \frac{24^2}{26^2} = \frac{26^2}{26^2} \)
	\item \( X^2 + Y^2 = Z^2 \) implies \( \left( \frac{X}{Z} \right)^2 + \left( \frac{Y}{Z} \right)^2 = 1 \)
\end{itemize}

\section*{Projective Co-ordinates}

\begin{itemize}
	\item Two-dimensional projective space \( P_K^2 \) over \( K \) is given by the equivalence classes of triples \( (x,y,z) \) with \( x, y, z \) in \( K \) and at least one of \( x, y, z \) non-zero.
	\item Two triples \( (x_1,y_1,z_1) \) and \( (x_2,y_2,z_2) \) are said to be equivalent if there exists a non-zero element \( \lambda \) in \( K \), such that:
	\[ (x_1,y_1,z_1) = (\lambda x_2, \lambda y_2, \lambda z_2) \]
	\item The equivalence class depends only on the ratios and hence is denoted by \( (x : y : z) \).
\end{itemize}

\section*{Singularity}

\begin{itemize}
	\item For an elliptic curve \( y^2 = f(x) \), define \( F(x,y) = y^2 - F(x) \). A singularity of the EC is a point \( (x_0,y_0) \) such that:
	\begin{itemize}
		\item \( \frac{\partial F}{\partial x}(x_0,y_0) = \frac{\partial F}{\partial y}(x_0,y_0) = 0 \)
		\item or, \( 2y_0 = -f'(x_0) = 0 \)
		\item or, \( f(x_0) = f'(x_0) \)
		\item Therefore, \( f \) has a double root.
	\end{itemize}
\end{itemize}

\section*{Singularity}

\begin{itemize}
	\item \( y^2 = x^2(x-1) \) double roots \( x=0 \)
	\item Let \( x-1=s^2 \)
	\item \( y^2 = (s^2+1)^2(s^2) \)
	\item Hence \( x=s^2+1, y=s(s^2+1) \)
\end{itemize}

\section*{If singular, then}

\begin{itemize}
	\item \( K \) = a field
	\item \( K(x, y) = K(t) \)
	\item For \( y^2 = x^2(x-1) \), \( x=s^2+1 \), \( y=s(s^2+1) \)
	\item For \( y^2 = x^3 \), \( y = t^3 \), \( x = t^2 \)
	\item For an elliptic curve, \( K(x, y) \) is never \( K(t) \).
\end{itemize}

\section*{Projective Form}

\begin{itemize}
	\item \( E: Y^2Z + a_1XYZ + a_3Y^2 = X^3 + a_2X^2Z + a_4XZ^2 + a_6Z^3 \)
	\item has a point \( (0, 1, 0) \), point at infinity, denoted by \( \mathcal{O} \).
\end{itemize}

\section*{Elliptic Curves in Characteristic 2}

\[ y^2 + a_1xy + a_3y = x^3 + a_2x^2 + a_4x + a_6 \]

\begin{itemize}
	\item If \( a_1 \) is not 0, this reduces to the form:
	\[ y^2 + xy = x^3 + Ax^2 + B \]
	\item If \( a_1 \) is 0, the reduced form is:
	\[ y^2 + a_3y = x^3 + Bx + C \]
	\item Note that the form cannot be:
	\[ y^2 = x^3 + Ax + B \]
\end{itemize}

\section*{EC over Finite Fields}

\begin{itemize}
	\item An elliptic curve may be defined over any finite field GF(q).
	\item For $GF(2^m)$, the curve has a different form:
	\[ y^2 + xy = x^3 + ax^2 + b \]
	\item where \( b \) is not 0.
	\item Addition formulae are similar to those over the reals.
\end{itemize}

\section*{Terminology}

\begin{itemize}
	\item Order of point \( P \) is the smallest integer \( r \) such that \( [r]P = \mathcal{O} \).
	\item Order of the curve is the number of points of \( E(\mathbb{F}) \), denoted by \( \#E(\mathbb{F}) \).
\end{itemize}

\section*{Group Properties}

\begin{itemize}
	\item Let \( \#E(\mathbb{F}_q) \) denote the number of points on an elliptic curve \( E(\mathbb{F}_q) \), including \( \mathcal{O} \).
	\item Hasse bound: \( \#E(\mathbb{F}_q) = q + 1 - t \), where \( |t| < 2 \sqrt{q} \).
	\item The group of points is either cyclic or a product of two cyclic groups.
\end{itemize}

\section*{So it's an Abelian Group...}

\begin{itemize}
	\item Group homomorphism? Isogeny, isogenous.
	\item Endomorphism, isomorphic.
	\item Examples of endomorphisms are:
	\begin{itemize}
		\item \( [2]: E \to E, P \mapsto [2]P \)
		\item \( [n]: E \to E, P \mapsto [n]P \)
	\end{itemize}
\end{itemize}

\section*{Non-trivial Isogeny}

\begin{itemize}
	\item \( E: y^2 = x^3 - x \)
	\item \( [i=\sqrt{-1}]: (x, y) \mapsto (-x, iy) \)
	\item \( [i=\sqrt{-1}]^2 = [i][i]=(-1): (x, y) \mapsto (x, -y), P \mapsto -P \)
	\item here \( i^2 = -1 \)
	\item \( 6^2 = -1 \mod 37 \)
	\item Called complex multiplication.
\end{itemize}

\section*{Frobenius Map}

\begin{itemize}
	\item GF(q), \( q = p^k \)
	\item \( F: \text{GF}(q) \to \text{GF}(q) \)
	\item \( F(x) = x^p \) for any \( x \)
	\item \( F \) is an isomorphism of GF(q). So \( F \) defines an isogeny for any elliptic curve over GF(q).
\end{itemize}

\section*{E[n]}

\begin{itemize}
	\item For any group \( G \), any natural number \( n \), \( G[n] = \{g | g^n = 1\} \).
	\item \( E[n] = \{P | [n]P = \mathcal{O}\} \).
\end{itemize}




	% Include more chapters as needed
	
	\appendix
	\chapter{Additional Data A}

\section{Existence of an Additional Root in Cubic Functions via the Intermediate Value Theorem}

\textbf{Theorem:} Let \( f(x) = ax^3 + bx^2 + cx + d \) be a cubic function, where \( a, b, c, d \in \mathbb{R} \) and \( a \neq 0 \). If \( x_1 \) and \( x_2 \) are two distinct roots of \( f(x) \), there exists at least one other root \( x_3 \) of \( f(x) \).

\textbf{Proof:}

\begin{enumerate}
	\item \textit{Cubic Function}: A cubic function is defined as \( f(x) = ax^3 + bx^2 + cx + d \), which is a polynomial of degree 3, and thus continuous over \(\mathbb{R}\).
	
	\item \textit{Known Roots}: Assume \( x_1 \) and \( x_2 \) are two distinct roots of \( f(x) \), i.e., \( f(x_1) = f(x_2) = 0 \).
	
	\item \textit{Intermediate Value Theorem (IVT)}: The IVT states that for any continuous function \( g \) on an interval \([a, b]\), if \( g(a) \) and \( g(b) \) have opposite signs, there is at least one \( c \) in \((a, b)\) such that \( g(c) = 0 \).
	
	\item \textit{Application to Cubic Function}: By the nature of cubic functions, they must change direction at least once between two roots. This implies the function will either attain a local maximum or minimum between \( x_1 \) and \( x_2 \).
	
	\item \textit{Existence of Third Root}: If the local extremum is above or below the x-axis, the function must cross the x-axis to change direction, implying the existence of another root \( x_3 \) in the interval \((x_1, x_2)\).
	
	\item \textit{Conclusion}: Therefore, by drawing a straight line through \( (x_1, 0) \) and \( (x_2, 0) \), this line will intersect the graph of \( f(x) \) at least at one other point, indicating the existence of another root \( x_3 \).
\end{enumerate}

\begin{tikzpicture}
	\begin{axis}[
		xlabel={$x$},
		ylabel={$f(x)$},
		axis lines=middle,
		xmin=-3, xmax=3,
		ymin=-3, ymax=3,
		grid=both,
		xtick={-2, 0, 2},
		ytick={-2, 0, 2},
		xticklabels={$x_1$, $0$, $x_2$},
		yticklabels={,,}
		]
		
		% Cubic function
		\addplot[smooth, thick, red] expression[domain=-2:2, samples=100]{x^3 - 3*x};
		
		% Dots at roots
		\addplot[only marks, mark=*, mark options={fill=blue}] coordinates {(-2,0) (2,0)};
		
		% Optional: Add a point for the third root
		\addplot[only marks, mark=*, mark options={fill=green}] coordinates {(0,0)};
		
		% Labels for the roots
		\node[below] at (axis cs:-2,0) {$x_1$};
		\node[below] at (axis cs: 2,0) {$x_2$};
		\node[above] at (axis cs: 0,0) {$x_3$};
		
	\end{axis}
\end{tikzpicture}

\begin{tikzpicture}
	\begin{axis}[
		xlabel={$x$},
		ylabel={$f(x)$},
		axis lines=middle,
		xmin=-3, xmax=3,
		ymin=-5, ymax=5,
		grid=both
		]
		
		% Define a cubic function
		\addplot[smooth, thick, red, domain=-3:3, samples=100] {x^3 - 3*x};
		
		% Mark the known roots
		\addplot[only marks, mark=*, mark options={fill=blue}] coordinates {(-sqrt(3),0) (sqrt(3),0)};
		
		% Draw a straight line through the two known roots
		\addplot[thick, blue, domain=-3:3] {0};
		
%		% Label for roots and potential third root
%		\node[below] at (axis cs:-sqrt(3),0) {$x_1$};
%		\node[below] at (axis cs: sqrt(3),0) {$x_2$};
%		\node[above] at (axis cs: 0,0) {Possible $x_3$};
		
	\end{axis}
\end{tikzpicture}

\begin{tikzpicture}[scale=0.5, >=Stealth]
	
	% Define the range for drawing
	\foreach \x in {-5,...,5} {
		\foreach \y in {-5,...,5} {
			% Place a dot at each point
			\node[draw,circle,inner sep=1pt,fill] at (\x,\y) {};
		}
	}
	
	% Draw vectors
	\draw[thick,->] (-4,-3) -- (4,-1) node[midway, below] {$\mathbf{v}_1$};
	\draw[thick,->] (-4,-3) -- (-3,1) node[midway, above] {$\mathbf{v}_2$};
	
	% Optional: Draw coordinate axes
	\draw[thin,->] (-6,0) -- (6,0) node[right] {$x$};
	\draw[thin,->] (0,-6) -- (0,6) node[above] {$y$};
	
\end{tikzpicture}

% Appendix A content

%	\chapter{Additional Data B}

\lipsum[9-10]  % Example text

% Appendix B content

	
	\backmatter
	% Bibliography, index, etc.
	
\end{document}