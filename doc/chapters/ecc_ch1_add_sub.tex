\chapter{NIST P-256}
\iffalse
\section{Time Measure in Linux}

\begin{lstlisting}[style=C]
#define _POSIX_C_SOURCE 200809L
#include <time.h>

typedef int8_t i8;
typedef int32_t i32;
typedef int64_t i64;

typedef uint8_t u8;
typedef uint32_t u32;
typedef uint64_t u64;

inline u64 rdtsc(void) {
	u32 lo, hi;
	
	__asm__ __volatile__ (
	"rdtsc" : "=a" (lo), "=d" (hi)
	);
	
	return ((u64)hi << 32) | lo;
}

u64 measure_cycle(void (*func)(u8*, u8*), u8* para1, u8* para2) {
	u64 start, end;
	const u64 num_runs = 10000;
	
	func(para1, para2);
	start = rdtsc();
	for (i32 i = 0; i < num_runs; i++)
		func(para1, para2);
	end = rdtsc();
	
	return (end - start) / num_runs;
}

double measure_time(void (*func)(u8*, u8*), u8* para1, u8* para2) {
	struct timespec start, end;
	double cpu_time_used;
	const double num_runs = 10000;
	
	func(para1, para2);
	clock_gettime(CLOCK_MONOTONIC, &start);
	for (i32 i = 0; i < num_runs; i++)
		func(para1, para2);
	clock_gettime(CLOCK_MONOTONIC, &end);
	
	cpu_time_used =
		(end.tv_sec - start.tv_sec) +
		(end.tv_nsec - start.tv_nsec) / 1e9;
	
	return cpu_time_used / num_runs;
}
\end{lstlisting}
\fi

\section*{Configuration}
\begin{lstlisting}[style=C]
#ifdef _WIN32 // Windows-specific definitions
#include <windows.h>
#include <stdint.h>
/* ... */
typedef DWORD       u32;
typedef DWORDLONG   u64;
#elif defined(__linux__) // Linux-specific definitions
#include <stdint.h>
/* ... */
typedef uint8_t     u8;
typedef uint32_t    u32;
typedef uint64_t    u64;
#else
#error "Unsupported platform"
#endif
// Define this to force 32-bit mode in development
#define FORCE_32_BIT
// Simplified check for 32-bit or forced 32-bit mode
#if defined(FORCE_32_BIT) || !defined(_WIN64) && !defined(__x86_64__)
#define IS_32_BIT_ENV
#endif

#ifdef IS_32_BIT_ENV // 32-bit specific settings
#define ONE     0x01
#define SIZE    8
typedef u32     word;
typedef word    field[SIZE];
#else // 64-bit specific settings
#define ONE     0x01LL
#define SIZE    4
typedef u64     word;
typedef word    field[SIZE];
#endif
\end{lstlisting}

\section{Data Representation}

\begin{table}[h!]\centering\renewcommand{\arraystretch}{1.25} % Increase row height by 1.5 times
	\begin{tabular}{@{\extracolsep{\fill}}>{\bfseries}l||c|c|c|c|c|c|c|c|c|c|c|c}
		\toprule[1.2pt]
		128-bit Hexa-string & \multicolumn{12}{c}{\texttt{0x77FDDC58464B01FC6606BC465BF5CBCB}} \\
		\hline
		String Index & \cellcolor{red!20}0 & \cellcolor{red!20}$\cdots$ & \cellcolor{red!20}7 & \cellcolor{green!20}8 & \cellcolor{green!20}$\cdots$ & \cellcolor{green!20}15 & \cellcolor{blue!20}16 & \cellcolor{blue!20}$\cdots$ & \cellcolor{blue!20}23 & \cellcolor{orange!20}24 & \cellcolor{orange!20}$\cdots$ & \cellcolor{orange!20}31 \\
		\hline
		\multirow{2}{*}{Split into Words} & \multicolumn{3}{c}{\texttt{77FDDC58}} & \multicolumn{3}{c}{\texttt{464B01FC}} & \multicolumn{3}{c}{\texttt{6606BC46}} & \multicolumn{3}{c}{\texttt{5BF5CBCB}} \\
		& \multicolumn{3}{c}{$\data[3]$} & \multicolumn{3}{c}{$\data[2]$} & \multicolumn{3}{c}{$\data[1]$} & \multicolumn{3}{c}{$\data[0]$}\\
		\hline
		\multirow{4}{*}{$\data[0]$} & \multicolumn{12}{c}{\texttt{(data[0] = '5'-'0';)} -> \texttt{(data[0]} $\ll=$ 4;)} \\
		& \multicolumn{12}{c}{-> \texttt{(data[0] |= 'B'-'0';)} -> \texttt{(data[0]} $\ll=$ 4;)}\\
		& \multicolumn{12}{c}{-> \texttt{(data[0] |= 'F'-'0';)} -> $\cdots$}\\
		\cline{2-13}
		& \multicolumn{12}{c}{\texttt{0x5 -> 0x50 -> 0x5B -> 0x5B0 -> $\cdots$}}\\
		\bottomrule[1.2pt]
	\end{tabular}
\end{table}
\begin{lstlisting}[style=C]
void stringToWord(word* wordArray, const char* hexString) {
	size_t length = strlen(hexString) / (SIZE == 8 ? 8 : 16);
	if (length != SIZE) {
		printf("Invaild 128-bit Hexa-string Length!\n");
		return;
	}
	for (size_t i = 0; i < length; i++)
#ifdef IS_32_BIT_ENV
		sscanf(&hexString[i * 8], "%8X", 
			&wordArray[(length - 1) - i]);
#else
		sscanf(&hexString[i * 16], "%16lX",
			&wordArray[(length - 1) - i]);
#endif
}
\end{lstlisting}
\begin{lstlisting}[style=C]
int main(void) {
	const char* string = "
		BD91C935C85617B079C6F2728B987CE4
		88BB17B4644D5F8B9C23AF955AB74663
	";
	
	word data[SIZE];
	stringToWord(data, string);
	
	for (i32 i = SIZE-1; i >=0; i--)
#ifdef IS_32_BIT_ENV
		printf("%8X:", data[i]);
#else
		printf("%16lX:", data[i]);
#endif
	puts("");
}
\end{lstlisting}
{\small\begin{lstlisting}[style=zsh]
BD91C935:C85617B0:79C6F272:8B987CE4:88BB17B4:644D5F8B:9C23AF95:5AB74663:
BD91C935C85617B0:79C6F2728B987CE4:88BB17B4644D5F8B:9C23AF955AB74663:
\end{lstlisting}}

\newpage
\begin{example}[secp256r1]
Consider $p = 2^{256} - 2^{224} + 2^{192} + 2^{96} - 1$.
\begin{table}[h!]\centering\renewcommand{\arraystretch}{1.25}
{\ttfamily\footnotesize\begin{tabular*}{\textwidth}{@{\extracolsep{\fill}}cccccccccc}
		$2^{256}$ & 00000001 & 00000000 & 00000000 & 00000000 & 00000000 & 00000000 & 00000000 & 00000000 & 00000000\\
		$-2^{224}$ & & 00000001 & 00000000 & 00000000 & 00000000 & 00000000 & 00000000 & 00000000 & 00000000 \\
		$+2^{192}$ & & & 00000001 & 00000000 & 00000000 & 00000000 & 00000000 & 00000000 & 00000000\\
		$+2^{96}$ & & & & & & 00000001 & 00000000 & 00000000 & 00000000 \\
		$-2^{0}$ & & & & & & & & & 00000001 \\ \cline{1-10}
		$p$ & & FFFFFFFF & 00000001 & 00000000 & 00000000 & 00000000 & FFFFFFFF & FFFFFFFF & FFFFFFFF \\
		$p_{\mathsf{inv}}$ & & 00000000 & FFFFFFFE & FFFFFFFF & FFFFFFFF & FFFFFFFF & 00000000 & 00000000 & 00000001 \\
\end{tabular*}}
\end{table}
\begin{lstlisting}[style=C]
#ifdef IS_32_BIT_ENV
static const field PRIME = {
	0xFFFFFFFFU, 0xFFFFFFFFU, 0xFFFFFFFFU, 0x00000000U,
	0x00000000U, 0x00000000U, 0x00000001U, 0xFFFFFFFFU
};
static const field PRIME_INVERSE = {
	0x00000001U, 0x00000000U, 0x00000000U, 0xFFFFFFFFU,
	0xFFFFFFFFU, 0xFFFFFFFFU, 0xFFFFFFFEU, 0x00000000U
};
#else
static const field PRIME = {
	0xFFFFFFFFFFFFFFFFU, 0x00000000FFFFFFFFU,
	0x0000000000000000U, 0xFFFFFFFF00000001U
};
static const field PRIME_INVERSE = {
	0x0000000000000000U, 0xFFFFFFFF00000000U,
	0xFFFFFFFFFFFFFFFFU, 0x00000000FFFFFFFFU
};
#endif
\end{lstlisting}
\end{example}

\section{secp256r1}
Define a prime number \begin{align*}
p_{256} &= 2^{256}-2^{224}+2^{192}+2^{96}-1 \\
&=2^{32*8}-2^{32*7}+2^{32*6}+2^{32*3}-2^{32*0}
\end{align*} that is used in the context of cryptography, particularly in the construction of elliptic curves for cryptographic purposes.
For prime $p$, let \begin{align*}
	m = \ceil*{\log_2 p},\quad t=\ceil*{\frac{m}{W}}.
\end{align*} For example, for $p=2^{256}$, we have $t=\ceil*{\frac{\ceil{\log_2 2^{256}}}{2^{32}}}$. Note that \begin{itemize}
\item $A = A[t-1]\adjacent\cdots\adjacent A[2]\adjacent A[1]\adjacent A[0]$
\item $a = 2^{(t-1)W}A[t-1]$+$\cdots 2^{2W}A[2]+2^{W}A[1]+A[0]$
\end{itemize}

\begin{table}[h!]\centering\renewcommand{\arraystretch}{1.25}
\begin{tabular}{C{.65cm}C{.65cm}C{.65cm}C{.65cm}|C{1.5cm}|C{.65cm}C{.65cm}C{.65cm}C{.65cm}|C{.65cm}C{.65cm}C{.65cm}C{.65cm}}
	\toprule[1.25pt]
	sign & $2^2$ & $2^1$ & $2^0$ & {Decimal} & \multicolumn{4}{c|}{{One's Complement}} & \multicolumn{4}{c}{{Two's Complement}} \\
	\midrule
	0 & 0 & 0 & 0 & +0 & 0 & 0 & 0 & 0 & 0 & 0 & 0 & 0 \\
	0 & 0 & 0 & 1 & +1 & 0 & 0 & 0 & 1 & 0 & 0 & 0 & 1 \\
	0 & 0 & 1 & 0 & +2 & 0 & 0 & 1 & 0 & 0 & 0 & 1 & 0 \\
	0 & 0 & 1 & 1 & +3 & 0 & 0 & 1 & 1 & 0 & 0 & 1 & 1 \\
	0 & 1 & 0 & 0 & +4 & 0 & 1 & 0 & 0 & 0 & 1 & 0 & 0 \\
	0 & 1 & 0 & 1 & +5 & 0 & 1 & 0 & 1 & 0 & 1 & 0 & 1 \\
	0 & 1 & 1 & 0 & +6 & 0 & 1 & 1 & 0 & 0 & 1 & 1 & 0 \\
	0 & 1 & 1 & 1 & +7 & 0 & 1 & 1 & 1 & 0 & 1 & 1 & 1 \\
	\hline\hline
	1 & 0 & 0 & 0 & -0 & 1 & 1 & 1 & 1 & 0 & 0 & 0 & 0 \\
	1 & 0 & 0 & 1 & -1 & 1 & 1 & 1 & 0 & 1 & 1 & 1 & 1 \\
	1 & 0 & 1 & 0 & -2 & 1 & 1 & 0 & 1 & 1 & 1 & 1 & 0 \\
	1 & 0 & 1 & 1 & -3 & 1 & 1 & 0 & 0 & 1 & 1 & 0 & 1 \\
	1 & 1 & 0 & 0 & -4 & 1 & 0 & 1 & 1 & 1 & 1 & 0 & 0 \\
	1 & 1 & 0 & 1 & -5 & 1 & 0 & 1 & 0 & 1 & 0 & 1 & 1 \\
	1 & 1 & 1 & 0 & -6 & 1 & 0 & 0 & 1 & 1 & 0 & 1 & 0 \\
	1 & 1 & 1 & 1 & -7 & 1 & 0 & 0 & 0 & 1 & 0 & 0 & 1 \\
	 &  &  &  & -8 &  &  &  &  & 1 & 0 & 0 & 0 \\
	\bottomrule[1.25pt]
\end{tabular}
\end{table}

\newpage

\vspace{4pt}
\begin{lstlisting}[style=C]
#ifdef _64BIT_SYSTEM
typedef u64 field_element[4]; // For 64-bit systems
#else
typedef u32 field_element[8]; // For 32-bit systems
#endif

// Example for modular addition (simplified)
void mod_add(field_element a, field_element b, field_element result) {
	uint64_t carry = 0;
	for (int i = 0; i < 4; ++i) { // Assuming 64-bit system
		uint64_t temp = (uint64_t)a[i] + b[i] + carry;
		result[i] = temp & 0xFFFFFFFFFFFFFFFF; // Keep only 64 bits
		carry = temp >> 64; // Carry for the next iteration
	}
	
	// Modular reduction if necessary
	if (carry || is_greater_or_equal(result, p256)) {
		// Subtract p256 if result >= p256
		subtract_p256(result);
	}
}

void subtract_p256(field_element x) {
	// This is a simplified version. In practice, you'd need to handle underflows.
	// Subtract (2^256 - 2^224 + 2^192 + 2^96 - 1)
	// In practice, implement this function based on the specific structure of p256
	// and considering the binary representation of the field elements.
}
\end{lstlisting}

\newpage
\section{Multi-Precision Addition}

\subsection{Theory for the Addition}

\begin{note}
A positive integer $X\in\intco{2^{w(n-1)}, 2^{wn}}$ is a $n$-word string. For example, let $w=32$ and consider $4$-word string
\begin{table}[h!]\centering\renewcommand{\arraystretch}{1.25}
\begin{tabular}{|c|c|c|c|}
\hline
$x[3]\cdot 2^{w*3}$ & $x[2]\cdot 2^{w*2}$ & $x[1]\cdot 2^{w*1}$ & $x[0]\cdot 2^{w*0}$\\
\hline
\end{tabular}
\end{table}\\
Then
\begin{table}[h!]\centering\renewcommand{\arraystretch}{1.25}
{\ttfamily\begin{tabular}{c|c|c|c|c|l}
\cline{2-5}
\textnormal{\bf Minimum} & 00000001 & 00000000 & 00000000 & 00000000 & $=2^{w*3}$\\ \cline{2-5}
\textnormal{\bf Maximum} & FFFFFFFF & FFFFFFFF & FFFFFFFF & FFFFFFFF & $=2^{w*4}-1$\\ \cline{2-5}
\end{tabular}}
\end{table}
\end{note}

\begin{tcolorbox}[colframe=procolor,title={\color{white}\bf Upper and Lower Bound of Addition}]
\begin{proposition}
	Let $w\in\set{32,64}$ be a word. Let $X$ and $Y$ are \(n\)-word and \(m\)-word strings, respectively, \ie, 
	\begin{align*}
		X &= x[n-1]\adjacent\cdots\adjacent x[0]\in\intco{2^{w(n-1)},2^{wn}}\quad\text{with}\quad x[n-1]\neq 0\\
		Y &= y[m-1]\adjacent\cdots\adjacent y[0]\in\intco{2^{w(m-1)},2^{mn}}\quad\text{with}\quad y[m-1]\neq 0\\
	\end{align*} Then \[
	{2}^{w\cdot(\max(n,m)-1)}<X+Y<2^{w\cdot(\max(m,n)+1)}.
	\]
\end{proposition}
\end{tcolorbox}
\begin{proof}
	Let $W:=2^w$. Then $X$ and $Y$ can be expressed as follows: 
	$\begin{cases}
		X=xW^{n-1}+X'\\
		Y=yW^{m-1}+Y'
	\end{cases}$ where \[
	a,b\in(0,W),\quad A'\in[0,W^{n-1}-1],\quad B'\in[0,W^{m-1}-1].
	\]
	Suppose that \(n\geq m\) then \begin{align*}
		W^{n-1}\leq\max(A,B)<A+B
		&=(aW^{n-1}+A')+(bW^{m-1}+B')\\
		&<(a+b)W^{n-1}+(W^{n-1}+W^{n-1})\\
		&=(a+b+2)W^{n-1}\\
		&\leq((W-1)+(W-1)+2)W^{n-1}\\
		&=2W^n\leq W^{n+1}.
	\end{align*} Thus $
	W^{n-1}<A+B<W^{n+1}.
	$ Here, \(n=\max(n,m)\).
\end{proof}

\begin{tcolorbox}[colframe=corcolor,title={\color{white}\bf }]
	\begin{corollary}
		\[
		\len_{\word}(X)=t=\len_{\word}(Y)\implies\len_{\word}(X+Y)\leq t+1.
		\]
	\end{corollary}
\end{tcolorbox}

\begin{tcolorbox}[colframe=procolor,title={\color{white}\bf Single-Word Addition \(x[i]+y[i]\)}]
\begin{proposition}
Let \(x,y\in\intco{0,2^w}\) are single-words. \begin{enumerate}[(1)]
	\item \(\len_\word(x+y)\leq 2\).
	\item (Division Theorem) \[
	x+y=q2^w+r=\floor*{\frac{x+y}{2^w}}\cdot 2^w+(x\boxplus y).
	\]
	\item (carry) \(\floor*{\frac{x+y}{2^w}}\in\set{0,1}\).
	\item $(\textcolor{red}{\star})$ $x\boxplus y < x\quad\Leftrightarrow\quad\floor*{\frac{x+y}{2^w}}=1$.
\end{enumerate}
\end{proposition}
\end{tcolorbox}
\begin{note}
\ \begin{table}[h!]\centering\renewcommand{\arraystretch}{1.25}
\begin{tabular}{c|c|c}\hline
	\multirow{2}{*}{$x\boxplus y < x$} & \texttt{True} & carry $=1$\\ \cline{2-3}
	& \texttt{False} & carry $=0$\\ \hline
\end{tabular}
\end{table}
\end{note}

\begin{tcolorbox}[colframe=procolor,title={\color{white}\bf Single-Word Addition with Carry \(x[i]\boxplus y[i]+\varepsilon\)}]
\begin{proposition}
Let \(x,y\in\intco{0,2^w}\) and \(\varepsilon\in\set{0,1}\). \begin{enumerate}[(1)]
	\item \(\len_\word((x\boxplus y)+\varepsilon)\leq 2\).
	\item (carry) \(\displaystyle\floor*{\frac{(x\boxplus y)+\varepsilon}{2^w}}\in\set{0,1}\).
	\item \[
	\floor*{\frac{x+y}{2^w}}=1
		\implies\floor*{\frac{(x\boxplus y)+\varepsilon}{2^w}}=0.
	\]
\end{enumerate}
\end{proposition}
\end{tcolorbox}
\begin{note}
\ \begin{table}[h!]\centering\renewcommand{\arraystretch}{1.25}
	\begin{tabular}{c|c|c|c|}\hline
		\multirow{2}{*}{$x\boxplus y < x$} & \texttt{True} & carry $=1$ & \multirow{2}{*}{$\implies$}\\ \cline{2-3}
		& \texttt{False} & carry $=0$ & \\ \hline
	\end{tabular}
\end{table}
\end{note}


\subsection{Practice for the Addition}

\begin{algorithm}[H]
\DontPrintSemicolon
\caption{Multi-Precision Addition}
\KwIn{$X,Y\intco{0,2^{wt}}\subseteq\Z$\tcp*{$(w,t)=(32,8)$ or $(w,t)=(64,4)$ for secp256r1}}
\KwOut{$(\varepsilon,Z)$ where $Z=(X+Y)\bmod 2^{wt}$ and $\varepsilon\in\set{0,1}$ is carry bit}
\BlankLine
\Fn{\textnormal{\Call{Addition\_Core}{$\varepsilon, Z, X, Y$}}}{
	$(\varepsilon,z[0])\gets x[0]+y[0]$\;
	\For{$i=1$ \KwTo $t-1$}{
		$(\varepsilon,z[i])\gets x[i] + y[i] + \varepsilon$\;
	}
	\Return $(\varepsilon, Z=z[t-1]\adjacent z[t-2]\adjacent\cdots\adjacent z[0])$\;
}
\end{algorithm}
\begin{example}
\ \begin{table}[h!]\centering\renewcommand{\arraystretch}{1.25}
	{\ttfamily\begin{tabular*}{\textwidth}{@{\extracolsep{\fill}}cccccc}
	$X$ & & 0xFFFFFFFF & 0xFFFFFFFF & 0xFFFFFFFF & 0xFFFFFFFF \\
	$Y$ & + & 0xFFFFFFFF & 0xFFFFFFFF & 0xFFFFFFFF & 0xFFFFFFFF \\
	$\varepsilon$ & 1 & 1 & 1 & 1 & 0 \\ \cline{2-6}
	$Z$ & 0x00000001 & 0xFFFFFFFF & 0xFFFFFFFF & 0xFFFFFFFF & 0xFFFFFFFE \\
	\end{tabular*}}
\end{table}
\end{example}

\begin{note}[How to Compute Carry?]
	content...
\end{note}

\begin{algorithm}[H]
	\DontPrintSemicolon
	\caption{Addition in $\F_p$}
	\KwIn{Modulus $p$, and integer $X,Y\in\intco{0,p}$}
	\KwOut{$Z = (X + Y) \bmod p$}
	\BlankLine
	$(\varepsilon,Z)\gets x[0]+y[0]$\;
	\For{$i=1$ \KwTo $t-1$}{
		$(\varepsilon,z[i])\gets x[i] + y[i] + \varepsilon$\;
	}
	\Return $(\varepsilon, Z=z[t-1]\adjacent z[t-2]\adjacent\cdots\adjacent z[0])$\;
\end{algorithm}
\begin{example}
\ \begin{table}[h!]\centering\renewcommand{\arraystretch}{1.25}
{\ttfamily\footnotesize\begin{tabular*}{\textwidth}{@{\extracolsep{\fill}}cccccccccc}
$\varepsilon$ & 1 & 1 & 1 & 1 & 1 & 1 & 1 & 0 & 0\\
$X$ & & BD91C935 & C85617B0 & 79C6F272 & 8B987CE4 & 88BB17B4 & 644D5F8B & 9C23AF95 & 5AB74663 \\
$Y$ & + & 4E272A73 & 41569559 & F3E58053 & BE961728 & D67BF71E & FBA44BF2 & 83DAA7ED & 9BF6DDA8 \\ \cline{1-10}
$Z$ & 1 & 0BB8F3A9 & 09ACAD0A & 6DAC72C6 & 4A2E940D & 5F370ED3 & 5FF1AB7E & 1FFE5782 & F6AE240B \\
$-p$ & & FFFFFFFF & 00000001 & 00000000 & 00000000 & 00000000 & FFFFFFFF & FFFFFFFF & FFFFFFFF \\ \cline{1-10}
$Z-p$ & & 0BB8F3AA & 09ACAD09 & 6DAC72C6 & 4A2E940D & 5F370ED2 & 5FF1AB7E & 1FFE5782 & F6AE240C
\end{tabular*}}
\end{table}\\
Note that
\begin{table}[h!]\centering\renewcommand{\arraystretch}{1.25}
{\ttfamily\footnotesize\begin{tabular*}{\textwidth}{@{\extracolsep{\fill}}cccccccccc}
$\varepsilon$ & 0 & 1 & 1 & 1 & 1 & 0 & 0 & 0 & 0 \\
$Z$ & 1 & 0BB8F3A9 & 09ACAD0A & 6DAC72C6 & 4A2E940D & 5F370ED3 & 5FF1AB7E & 1FFE5782 & F6AE240B \\
$+p_{\mathsf{inv}}$ & & 00000000 & FFFFFFFE & FFFFFFFF & FFFFFFFF & FFFFFFFF & 00000000 & 00000000 & 00000001 \\ \cline{1-10}
$Z+p_{\mathsf{inv}}$ & 1 & 0BB8F3AA & 09ACAD09 & 6DAC72C6 & 4A2E940D & 5F370ED2 & 5FF1AB7E & 1FFE5782 & F6AE240C
\end{tabular*}}
\end{table}
\end{example}

\newpage
\section{Multi-Precision Subtraction}
\subsection{Theory for the Subtraction}
\begin{tcolorbox}[colframe=procolor,title={\color{white}\bf Upper and Lower Bound of Subtraction}]
\begin{proposition}
Let $w\in\set{32,64}$ be a word. Let $X$ and $Y$ are \(n\)-word and \(m\)-word strings, respectively, \ie, 
\begin{align*}
	X &= x[n-1]\adjacent\cdots\adjacent x[0]\in\intco{2^{w(n-1)},2^{wn}}\quad\text{with}\quad x[n-1]\neq 0\\
	Y &= y[m-1]\adjacent\cdots\adjacent y[0]\in\intco{2^{w(m-1)},2^{wm}}\quad\text{with}\quad y[m-1]\neq 0\\
\end{align*} Then, for $X\geq Y$, \[
0\leq X-Y<X<2^{wn}.
\]
\end{proposition}
\end{tcolorbox}

\subsection{Practice for the Subtraction}

\begin{algorithm}[H]
	\DontPrintSemicolon
	\caption{Multi-Precision Subtraction}
	\KwIn{$X,Y\intco{0,2^{wt}}\subseteq\Z$\tcp*{$(w,t)=(32,8)$ or $(w,t)=(64,4)$ for secp256r1}}
	\KwOut{$(\varepsilon,Z)$ where $Z=(X-Y)\bmod 2^{wt}$ and $\varepsilon\in\set{0,1}$ is borrow bit}
	\BlankLine
	\Fn{\textnormal{\Call{Addition\_Core}{$\varepsilon, Z, X, Y$}}}{
		$(\varepsilon,z[0])\gets x[0]-y[0]$\;
		\For{$i=1$ \KwTo $t-1$}{
			$(\varepsilon,z[i])\gets x[i] - y[i] - \varepsilon$\;
		}
		\Return $(\varepsilon, Z=z[t-1]\adjacent z[t-2]\adjacent\cdots\adjacent z[0])$\;
	}
\end{algorithm}
\begin{example}
\ \begin{table}[h!]\centering\renewcommand{\arraystretch}{1.25}
{\ttfamily\begin{tabular*}{\textwidth}{@{\extracolsep{\fill}}cccccc}
$X$ & & 0x00000000 & 0x00000000 & 0x00000000 & 0x00000000 \\
$-Y$ &  & 0xFFFFFFFF & 0xFFFFFFFF & 0xFFFFFFFF & 0xFFFFFFFF \\
$-\varepsilon$ & 1 & 1 & 1 & 1 & 0 \\ \cline{2-6}
$Z$ & 0x00000001 & 0x00000000 & 0x00000000 & 0x00000000 & 0x00000001 \\
\end{tabular*}}
\end{table}
\end{example}
